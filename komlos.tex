\documentclass[anon,12pt]{colt2019} % Anonymized submission
% \documentclass[12pt]{colt2019} % Include author names

% The following packages will be automatically loaded:
% amsmath, amssymb, natbib, graphicx, url, algorithm2e

\title[Discrepancy, Coresets, and Sketches in Machine Learning]{Discrepancy, Coresets, and Sketches in Machine Learning}
\usepackage{times}
% Use \Name{Author Name} to specify the name.
% If the surname contains spaces, enclose the surname
% in braces, e.g. \Name{John {Smith Jones}} similarly
% if the name has a "von" part, e.g \Name{Jane {de Winter}}.
% If the first letter in the forenames is a diacritic
% enclose the diacritic in braces, e.g. \Name{{\'E}louise Smith}

% Two authors with the same address
% \coltauthor{\Name{Author Name1} \Email{abc@sample.com}\and
%  \Name{Author Name2} \Email{xyz@sample.com}\\
%  \addr Address}

% Three or more authors with the same address:
% \coltauthor{\Name{Author Name1} \Email{an1@sample.com}\\
%  \Name{Author Name2} \Email{an2@sample.com}\\
%  \Name{Author Name3} \Email{an3@sample.com}\\
%  \addr Address}

\usepackage{amsmath}
%\usepackage{amsthm}
\usepackage[noend]{algorithmic}
\usepackage{algorithm}
%\usepackage[margin=1in]{geometry}
\usepackage{xcolor}
\usepackage{amsfonts}
\usepackage{ amssymb }

%\newtheorem{theorem}{Theorem}[section]
%\newtheorem{conjecture}{Conjecture}[section]
%\newtheorem{claim}[theorem]{Claim}
%\newtheorem{lemma}[theorem]{Lemma}
%\newtheorem{remark}[theorem]{Remark}
%\newtheorem{definition}[theorem]{Definition}
%\newtheorem{observation}[theorem]{Observation}
%\newtheorem{corollary}[theorem]{Corollary}
%\newtheorem{proposition}[theorem]{Proposition}
\newtheorem{fact}[theorem]{Fact}
%\newtheorem*{definition*}{Definition}
%\newtheorem{definition}{Definition}
%\newtheorem{claim}{Claim}
%\newtheorem{theorem}{Theorem}


\newcommand{\zk}[1]{\textcolor{red}{ZK: #1}}
\newcommand{\el}[1]{\textcolor{blue}{EL: #1}}

\newcommand{\ip}[1]{\left \langle #1 \right \rangle}
\newcommand{\R}{\mathbb{R}}
\newcommand{\E}{\mathbb{E}}
\newcommand{\D}{\rho}
\newcommand{\eps}{\epsilon}
\newcommand{\F}{\mathcal{F}}
\newcommand{\X}{\mathcal{X}}
\newcommand{\Q}{\mathcal{Q}}
%\usepackage{ntheorem}
\newcommand{\floor}[1]{\left \lfloor #1 \right \rfloor}
\newcommand{\ceil}[1]{\left \lceil #1 \right \rceil}
\newcommand{\disc}{\text{disc}}
\renewcommand{\Pr}{\operatorname{Pr}}

\newcommand{\komconstant}{c}


% Authors with different addresses:
\coltauthor{%
 \Name{Zohar Karnin} \Email{zkarnin@amazon.com}%\\ \addr Address 1
 \AND
 \Name{Edo Liberty} \Email{libertye@amazon.com}%\\ \addr Address 2%
}


\begin{document}

\maketitle

\begin{abstract}
This paper defines the Komlos complexity for families of functions and conjectures that it governs minimal coreset sizes for different problems in machine learning. Our result resolves a long standing open question regarding minimal cardinality coresets for kernel density estimation. It provides improved coreset for matrix covariance approximation, logistic regression, and any analytic function of the the dot product. The results leverage new tools and algorithms from discrepancy theory and provide tools for improving coreset sizes for other problems. Independently of the above, we show that any problem with low Komlos complexity admits a small merge-able streaming sketch. 
The result is an exponential improvement over the widely known and simple merge-and-reduce trick.
This paper makes an explicit connection between low discrepancy, coresets, learnability, and streaming algorithms. 
\end{abstract}


\section{Introduction}
In both machine learning and in streaming and sketching problems our goal is (usually) to approximate the sums or expectations of functions.
Specifically, approximate $F = \sum_{i=1}^{n} f(x_i)$ for every $f\in \F$ where $\F$ is a family of functions and $x_i \in \mathcal X$ are either training examples or stream items.
In this paper it will often be convenient to consider an equivalent formulation.
$F(q) = \sum_{i=1}^{n} f(x_i, q)$  and $q \in \mathcal Q$ is either the model parameters (which acts as an ``index'' into $\F$) or a query from a possible set $\Q$. 
Clearly, the two formulations are identical. In this manuscript we use both interchangeably.  
For simplicity and throughout this manuscript we assume the range of $f$ is $[0,1]$.
The goal is to produce a coreset. This is a set of weights $w$ with at most $k \ll n$ non-zeros such that $\sum_{i=1}^{n}w_i f(x_i,q) := \tilde F(q)$ approximates $F(q)$.
Approximation here means that $|\tilde F(q)  - F(q)| \le \eps n$ either for all possible queries $q \in \mathcal Q$ simultaneously or for every fixed $q$ with probability at least $1-\delta$. 
There are more complicated formulations such as weak coresets which we will not touch upon in this manuscript.

Generating a concise representation $\tilde F$ for $F$ allows one to optimize over $\tilde F$ instead of $F$ which is more efficient.  Moreover, if the resulting sketches or coresets are mergable, this could be done on separate machines without the need for communication or assuming randomness in the partitioning.


For bounded functions $f$ like above, uniform sampling of $\ell = O(\log(1/\delta)/\eps^2)$ combined with a union bound over $|\mathcal Q|$ always provides a valid solution using $\ell = O(\log(|\mathcal Q|)/\eps^2)$ items. 
While $|\mathcal Q|$ is often infinite it reduces to a finite (albeit usually exponentially large) set through standard epsilon net arguments. 
%
We argue that we can offer a solution for producing smaller coresets for all of the above problems in a unified manner. 
Moreover, our solution creates streaming algorithms with fully mergeable sketches. 
The size of the coreset above appears to be intimately tied to the discrepancy properties of the associated functions.
This is captured by its Komlos complexity.  


\section{Komlos Complexity}
We begin by formally defining the Komols complexity and by giving some general results about for problems with low Komols complexity. The Komols complexity of a sequence set is defined similarly to it Rademacher complexity.

\begin{definition}
Let $\F$ be a family of functions $f:\X\rightarrow\R$ and $\sigma \in \{-1,1\}^m$ uniformly at random. 
The Rademacher complexity of $\F$ w.r.t.\ $\{x_1,\ldots,x_m\} \subset \X$ is 
\[
R_m(\F) =  \frac{1}{m}\E_{\sigma} \sup_{f \in \F}  \left| \sum_{i=1}^{m}\sigma_i f(x_i) \right|
\]
\end{definition}
%
In other words, the Radamacher complexity of a family function $\F$ measures how much a function in the family can skew the values when weighted with random signs. The Komlos complexity measures the ability of the family function to skew an adversarial set of signs.

\begin{definition}
Let $\F$ be a family of functions $f:\X\rightarrow\R$ and $\sigma \in \{-1,1\}^m$. 
The Komlos complexity of $\F$ w.r.t.\ $\{x_1,\ldots,x_m\} \subset \X$ is 
\[
K_m(\F) =  \frac{1}{m}\min_{\sigma} \sup_{f \in \F}  \left|\sum_{i=1}^{m}\sigma_i f(x_i)\right|
\]
The Komlos Complexity $K_m(\F)$ without a reference set $\{x_1,\ldots,x_m\}$ is the upper bound on any such set.
\end{definition}


\noindent It well known that $R_m$ is a good measure for generalization when data is chosen uniformly at random (or drawn from an unknown distribution). See \cite{Bartlett:2003:RGC:944919.944944} for other notions of complexity and relationships between them. We claim that $K_m$ is the measure more suitable for creating coresets, and for creating mergeable sketches.


\el{Add a much longer explanation about the connection to discrepancy and the ideas behind this construction} The name Komlos Complexity is due to the Komlos Conjecture. It claims that $\min_\sigma \max_f \|\sum_i \sigma_i f_i\|_\infty = O(1)$ for unit vectors $f$. The correctness of the Komlos Conjecture is still a fundamental open problem in the study of discrepancy theory.  The definition is useful, however, regardless of the correctness or provability of the conjecture.  
It is obvious that  $K_m(\F) \le R_m(\F)$ for any family $\F$. We show later that $K_m(\F) = o(R_m(\F))$ for a wide range of interesting problems in machine learning.

 \zk{I actually am still having problems with justifying the name Komlos complexity. It might as well be called Banaszczyk's complexity or simply discrepancy complexity. In particular if you could prove Komlos' conjecture it wouldn't affect any of our results, nor simplify our proofs.. Either way, below is my suggested motivation for the definition}
 \el{You are right about the name. In fact, I've been thinking about it myself. The discrepancy we get is actually much lower than what Komlos conjecture says... So, I am happy with with calling it simple \it{the discrepancy.}}
To understand the motivation of the the Komlos Complexity, consider the following informal explanation of Radamacher Complexity applied to ML problems. In this ML problem there exists a distribution over $(x,y)$ pairs where we aim to predict $y$ given $x$. Our goal is to find a predictor from a family of prediction functions that suffers the least loss. A predictor maps to a loss value on each $x,y$ pair so we can discuss the problem of finding a function $f$ from a family $\F$ minimizing $\E_{x,y} f(x,y)$ where $f$ is the loss associated with the corresponding predictor on the $x,y$ pairs. Having a low Radamacher complexity means that if we sample $m$ pairs, split them into two uniformly at random and find the function $f$ minimizing the loss over the first half, the loss over the second half is bounded. This translates to a generalization bound stating that the function $f$ minimizing the loss over a uniform random sample of $(x,y)$ pairs, will perform well on the entire distribution. In other words, the Radamacher complexity gives a guarantee for the distribution loss of the minimizer of a finite set of pairs chosen uniformly at random.

Coming back to Komlos Complexity, having the ability to choose the signs in a non-uniform way lets us choose a subset of pairs $(x,y)$ such that the minimizer $f$ over the subset is guaranteed to have (roughly) the same performance on the entire set. This allows us to obtain a guarantee for the distribution loss of a minimizer of a finite set of pairs chosen in an adaptive way to the family function $\F$. This set is in fact a core-set and the Komlos complexity helps us determine that possible size of a coreset. The tradeoff of considering a coreset vs.\ a random sample is that on one hand, it is clearly much simpler to obtain a random sample. On the other hand we will show in the following sections several examples for which a coreset can be significantly smaller than the random sample while maintaining the same guarantees. This will be done by showing that for a wide range of interesting problems in machine learning $K_m(\F) = o(R_m(\F))$.
\el{I like this explanation, I'm just not sure why it talks about pairs...}

\subsection{General Framework for Coresets}

In what follows it will be convenient to represent the function family as a family of vectors representing the functions. The vector family will be denoted as $Q$, where we use $q$ to denote a single vector corresponding to a function. We a universal function $f$ such that $f(x,q)$ corresponds to the value of the function matching $q$ when applied to $x$. We will assume from hereon that for any $x,q$ $|f(x,q)| \leq 1$; this allows to simplify the presentation. Our results can extend to $|f(x,q)| \leq R$ and will have dependencies on the bound $R$. Achieving these bounds is a purely technical task that we defer for the sake of readability. 


Consider the function $F(q) = \sum_{i=1}^{m} f(x_i,q)$ and the signed sum of error function $E(q) = \sum_{i=1}^{m} \sigma_i f(x_i,q)$ where $\sigma_i \in \{-1,1\}$.
Now consider $F_1(q) = F(q) + E(q) = \sum_{i=1}^{m} f(x_i,q)  + \sum_{i=1}^{m} \sigma_i f(x_i,q)  = 2 \sum_{i | \sigma_i=1} f(x_i,q)$ and similarly $F_{-1}(q)$. We have that both $F_1(q), F_{-1}(q)$ are approximations for $F(q)$ obtained by coresets of item weights of $2$ and an error at most $|F_1(q) - F(q)| =  |E(q)|$.

%
The above is true for any choice of signs $\sigma_i$, specifically, for those minimizing $\max_q | E(q)|$.
Note that we can select signs such that $|E(q)| \le m K_m(\F)$.
A function is sketch-able if $K_m(\F) = o(m)$.
Typically we see that $K_m(\F) = O(c/m)$ or $K_m(\F)= c/\sqrt{m}$ for some $c$ which does not depend on $m$.

\begin{fact}
Let $\F$ be a function family with a corresponding Komlos Complexity $K_m(\F) = O(c/m)$. Let $F(q) = \sum_{i=1}^{n}f_i(q)$ where $f_i \in \F$. There exists a coreset for $F$ of size $k = O(c/\eps)$ whose error is at most $\eps n$.
\end{fact}
\begin{fact}

For any function family $\F$ with a corresponding Komlos Complexity $K_m(\F) = O(c/\sqrt{m})$ there exists a coreset of size 
$O(c^2/\eps^2)$ whose error is at most $\eps n$.

\end{fact}

\noindent Proving both facts is trivial and known in the folklore as ``the halving trick". 
At step one, create a coreset of size at most $n/2$ of items of weight $2$ and incur an error of at most $c$ or $c \sqrt{n}$.
Then, we repeat the process and create a coreset of size $n/4$ of items of weight $4$. Here, you incur error of $2c$ or $2c\sqrt{n/2}  = c\sqrt{2n}$.
Note that the sum of errors is a geometric sequence that is asymptotically dominated by its last element. 
Halting the compression at $\Theta(c/\eps)$ or $\Theta(c^2/\eps^2)$ items respectively achieves the goal.



\subsection{General Framework for Sketching}\label{sec:sketch}
We claim that low Komlos complexity implies concise streaming mergeable coresets. 
This section assumes the existence of an algorithm for finding low discrepancy signs $\sigma$.

\begin{theorem} \label{thm:streaming}
For any function family $\F$ with a corresponding Komlos Complexity with $K_m = O(c/m)$ there exists an fully-mergeable streaming coreset algorithms of size 
$$O\left(c\log^2\left(\min(\eps n/c, \log(n/\delta)) \right)/\eps\right)$$ 
whose error is at most $\eps n$ and failure probability at most $\delta$. For $\delta=0$ this results in a deterministic algorithm. The algorithm must be given $\eps, \delta$ as input and does not require a priori knowledge of the stream length $n$.
\end{theorem}

\begin{theorem} \label{thm:streaming2}
For any function family $\F$ with a corresponding Komlos Complexity of $K_m = O(c/\sqrt{m})$ there exists an unbiased fully-mergeable streaming coreset algorithms of size 
$$O\left(c^2\log^3\left(\min(\eps^2 n/c, \log(n/\delta)) \right)/\eps^2\right)$$ 
whose error is at most $\eps n$ and failure probability at most $\delta$. For $\delta=0$ this results in a deterministic algorithm. The algorithm must be given $\eps, \delta$ as input and does not require a priori knowledge of the stream length $n$.
\end{theorem}

The proofs of the above two theorems are quite technical and not very illuminating. Most of the ideas are already present in the recent breakthrough work on streaming quantiles \cite{DBLP:conf/focs/KarninLL16}. 


\section{The Complexity of Analytic Functions of Dot Products} \label{sec:analytic}

Now that we proved the usefulness of the Komlos Complexity we move to upper bound it for interesting family functions. We provide a coreset suitable for analytical functions of the inner product $\ip{q,x}$ or Euclidean distance $\|q-x\|$. 

The idea is to find a set of signs that simultaneously balance $\ip{q,x}^k$ for all powers $k$ and unit vectors\footnote{We Assume that $\|x\|,\|q\| \leq 1$ for ease of presentation. As above our results extend to generic bounds on the radius of $q$} $q$. By controlling all powers of $\ip{q,x}$ we control any sum of these powers. It follows that this coreset can be used to control, for example, the logistic loss function $L(q,x) = \log(1+\exp(\ip{q,x}))$ or the gaussian Kernel $K(q,x) = \exp(-\lambda \|q-x\|^2)$. 


We start with some notation and trivial properties. 
For a vector $q \in \R^d$ let $q^{\otimes k}$ represent the $k$-dimensional tensor obtained from the outer product of $q$ with itself $k$ times. For a $k$ dimensional tensor with $d^k$ entries $X$ we consider the measure
$\|X\|_{T_k} = \max_{q \in \R^d, \|q\|=1} \left| \langle X, q^{\otimes k}\rangle \right|$.
\begin{fact}
$\|X\|_{T_k}$ is a norm
\end{fact}
\begin{proof}
We prove the claim directly from the definition of a norm.
Notice that for any $X \neq 0$, $\ip{X, q^{\otimes}}$ is a non-zero polynomial in $q$. It follows that there must be $q$ for which its value is non-zero, meaning that $\|X\|_{T_k}=0$ iff $X=0$. For a scalar $a$, we clearly have by definition that
$\|aX\|_{T_k} = |a|\|X\|_{T_k}$.  Lastly, by the max definition we  have
$ \|X+Y\|_{T_k} =  \max_q \left| \langle X+Y, q^{\otimes k}\rangle \right| \leq 
\max_q \left| \langle X, q^{\otimes k}\rangle \right| + \max_q\left| \langle Y, q^{\otimes k}\rangle \right| = \|X\|_{T_k} + \|Y\|_{T_k}$
\end{proof}

We are now ready for the lemma controlling all powers of inner products simultaneously. 

\begin{lemma}\label{uc}
For any set of vectors $x_i \in \R^d$ with $\|x_i\| \leq 1$ there exist a set of signs $\sigma_i$ such that for all $k$ simultaneously $\left\| \sum_i \sigma_i x_i^{\otimes k} \right\|_{T_k} \le O(\sqrt{d k\log^{3}{k}})$ (the $3$ power of the term $\log(k)$ can be reduced to any constant power larger than $2$). 
\end{lemma}
\begin{proof}
The proof will use Banaszczyk's theorem \cite{Banaszczyk}. 
Let $\mathcal K$ be a convex body in Euclidean space with Gaussian measure at least 1/2 ($\Pr[g \in \mathcal K] \ge 1/2$ when $g$ is i.i.d.\ Gaussian).
Let $x_1,\ldots,x_n$ be vectors with $\|x_i\| \leq 1$. 
Then, there exist signs $\sigma_i$ such that $\sum \sigma_i x_i \in C \mathcal K$ for some constant $C$.

To use Banaszczyk's theorem we begin with defining our convex body.
Define the norm $\|\psi\|_T$ of a vector $\psi$ as follows. Look at the first $d$ coordinates of $\psi$ as a vector $\psi_1$, the next $d^2$ coordinates of $\psi$ as a matrix $\psi_2$ the next $d^3$ coordinates as a three tensor $\psi_3$ etc.
We define $\|\psi\|_T = \max_k \|\psi_k\|_{T_k} /\sqrt{\log(k)}$. 
Here, $\|\cdot\|_{T_k}$ is the special spectral norm defined in the beginning of the section 
The maximum over norms of subvectors is clearly a norm in itself, meaning that $\|\cdot \|_T$ is indeed a norm. It follows that  the set $\mathcal K  = \{\psi \; | \; \|\psi\|_T \le c\sqrt{d}\}$ is convex. 

We now need to show that the Gaussian measure of $\mathcal K$ is at least $1/2$. 
That is, with probability at least $1/2$ a vector of random Gaussian entrees $g$ belongs to $\mathcal K$.
Consider a random i.i.d.\ Gaussian Tensor $g_k \in \R^{d^k}$. 

A trivial modification of Theorem 1 from \cite{tomioka2014spectral} shows that $\Pr[\|g_k\|_{T_k} \ge c\sqrt{d\log(k)}] \le 1/10k^2$ for some constant $c$. The only change needed in the proof is the size of the epsilon net which changes from $(2\log(3/2)/k)^{kd}$ for \cite{tomioka2014spectral} to $(2\log(3/2)/k)^d$, where the reason we require a net over a smaller space is due to us bounding the inner product with a rank one tensor rather than rank $k$. Union bounding on all values of $k$ we get $\sum_k 1/10k^2 \le 1/2$ which shows $g = [g_1, \operatorname{flat}(g_2), \operatorname{flat}(g_3), \ldots]$ belongs to $\mathcal K$ with probability at least $1/2$, where $\operatorname{flat}(g_k)$ is the flattening of the tensor into a one dimensional vector. 
%
We now define a mapping $\psi(x)$ of $x\in \R^d$ to a high dimensional space. The function $\operatorname{flat}(X)$ simply strings the entry values of the tensor $X$ into a vector.
$$\psi(x) = \left[x, \frac{\operatorname{flat}(x^{\otimes 2})}{\sqrt{2\log^2(2)}}, \frac{\operatorname{flat}(x^{\otimes 3})}{\sqrt{3\log^2(3)}}, \ldots,\frac{\operatorname{flat}(x^{\otimes k})}{\sqrt{k\log^2(k)}},\ldots \right]$$

Finally, note that for $\|x\| \le 1$ we have $\|\psi(x)\|_2 = (\sum_k  1/k\log^2(k))^{1/2} = O(1)$.


We are now ready to apply Banaszczyk's theorem. 
There exist signs $\sigma_i$ such that $\psi  = \sum_i \sigma_i \psi(x_i) \in C \mathcal K$.
Since $\psi_k = \sum_i \sigma_i x_i^{\otimes k}/\sqrt{k \log^2{k}}$ we get that 
$$\max_k \frac{\|\sum_i \sigma_i  x_i^{\otimes k}\|_{T_k}}{\sqrt{k \log^{3}(k)}} \le O\left( \sqrt{d} \right)$$
\end{proof}

\begin{lemma} \label{lem:komlos anl}
Let $f$ be a function of $\langle x,q\rangle$ such that $f(\langle x,q\rangle) = \sum_k \alpha_k \langle x,q\rangle^k$ is its Taylor expansion. The Komlos Complexity of the Family of functions $f_q(x) = f(\ip{q,x})$, indexed by $\|q\| \leq 1$, is bounded by
\[
K_m = \min_\sigma \sum_i \sigma_i f(x_i,q) =O\left( \sqrt{d} \sum_k  |\alpha_k|\sqrt{k\log^3(k)}\right)
\]
For general $\|q\| \leq R$ we get
\[
K_m = \min_\sigma \sum_i \sigma_i f(x_i,q) =O\left( \sqrt{d} \sum_k  |\alpha_k| R^k \sqrt{ k\log^3(k)}\right)
\]
\end{lemma}
\begin{proof}
The proof follows from combining the above.
$$
\sum_i \sigma_i f(x_i,q) = \sum_k \alpha_k \sum_i \sigma_i \ip{ x_i,q}^k =  \sum_k \alpha_k  \ip{  \sum_i \sigma_i x_i^{\otimes k},q^{\otimes k}} \le $$
$$\sum_k |\alpha_k| \cdot \left\| \sum_i \sigma_i x_i^{\otimes k}\right\|_{T_k} \cdot \|q\|^k
$$
By Lemma \ref{uc} we can find signs $\sigma$ such that 
$\left\| \sum_i \sigma_i x_i^{\otimes k}\right\|_{T_k} \le c\sqrt{d k \log^3(k)}$. Substituting into the above, the lemma follows.
\end{proof}

\begin{theorem}\label{analitic1}
Let $f:\R\rightarrow\R$ be analytic. There exist a radius $R$ such that the function family of functions $f_q(x) = f(\ip{q,x})$, indexed by $\|q\| \leq R$, has Komlos complexity of $O(\sqrt{d}/m)$. 
\end{theorem}
\begin{proof}
Recall that for analytic functions $f$ we have $\left| \frac{d^k f}{dz^k}(z) \right|  \leq C^{k+1} k! $
for some constant $C$. Considering the taylor expansion of $f$ near zero, for $R < 1/C$ the sum
$ \sum_k  |\alpha_k| R^k \sqrt{ k\log^3(k)} \leq C \sum_k  (CR)^k \sqrt{ k\log^3(k)}$
corresponding to Lemma~\ref{lem:komlos anl} converges to a constant. The result follows.
\end{proof}

\begin{corollary}
The Komlos complexity of the Logistic function $f(\ip{q,x}) = \log(1+\exp(\ip{q,x}))$ in dimension $d$, for $\|q\| \leq 1$ is $O(\sqrt{d}/m)$.
\end{corollary}

\begin{corollary}
The Komlos complexity of the covariance function $f(\ip{q,x}) = \ip{q,x}^2$ in dimension $d$, for $\|q\| \leq 1$ is $O(\sqrt{d}/m)$. This gives coresets for matrix column subset selection such that $\|XX^T - \tilde X \tilde X^T\| \le \eps n$ where $\tilde X$ contains $O(\sqrt{d}/\eps)$ rescales columns of the matrix $X$.
\end{corollary}



\begin{theorem} \label{thm:analytic2}
Let $f:\R\rightarrow\R$ be analytic. There exist a radius $R$ such that the function family of functions $f_q(x) = f(\|x-q\|^2)$, indexed by $\|q\| \leq R$, has Komlos complexity of $O(\sqrt{d}/m)$. 
\end{theorem}
\begin{proof}
By transforming $x$ to $\tilde{x} = (1, \sqrt{2}x, \|x\|^2)$ and $q$ to $\tilde{q} = (\|q\|^2, -\sqrt{2}q, 1)$ we get $\ip{\tilde{x},\tilde{q}} = \|q-x\|^2$. Moreover, $\|q\| \le R$ gives $\|\tilde q\| \le R^2+1$. The result follows from applying Theorem~\ref{analitic1} to $f(\ip{ \tilde q, \tilde x}) = f(\|q-x\|^2)$.
\end{proof}

\begin{corollary}
The Komlos complexity of the Gaussian kernel $K(q,x) = \exp(-\gamma \|x-q\|^2)$ in dimension $d$ is $O((1+\gamma)\exp(\gamma)\sqrt{d}/m)$.
\end{corollary} 
This improves upon the recent result of \cite{DBLP:journals/corr/abs-1802-01751} by proving the existence of $\eps$ approximation corsets of size $\sqrt{d}/\eps$ for Gaussian kernel density, in the case where $\gamma$ is constant. For the constant $\gamma$ setting this resolves the open problem raised by \cite{DBLP:journals/corr/abs-1802-01751} and matches their lower bound.   

\begin{proof}
W.l.o.g. the maximum distance $\|q-x\|$ is 1. The taylor series of $K$ becomes
$$ \sum_{k=0}^\infty \frac{(-\gamma)^k}{k!} $$
Plugging into the equation in the proof of Theorem~\ref{thm:analytic2} we get that the sum determining the constant is upper bounded by
$$ \sum_{k=0}^\infty \frac{\gamma^{k}\sqrt{ k\log^3(k)}}{k!} = O\left((1+\gamma) \exp(\gamma)\right)$$
\end{proof}
%\end{proof}


\section{A Simple Algorithm for Kernel Density Estimation}
From section \ref{sec:analytic} we know that the Komlos complexity of the Gaussian kernel is $K_m = O(\sqrt{d}/m)$. The result however is non-constructive in the sense that we are not aware of an algorithm that can obtain the $\sigma$ signs efficiently. 
Here, we provide a computationally efficient bound that can be achieved with a straightforward algorithm of complexity $O(m^2)$. Together with the results of Section~\ref{sec:sketch} this provides an efficient sketching algorithm for Kernel Density Estimation. We show that for any positive kernel $K_m = O(1/\sqrt{m})$. This bound is superior to that of the previous section when $d > \sqrt{m}$. More importantly though, there is a very simple, intuitive, and deterministic algorithm for computing the signs $\sigma$. 
Given a collection of data points $X = x_1,\ldots, x_n$ in $\R^d$ the density function $f: \R^d \rightarrow \R$ of a point $q$ is defined as 
$$ f(q) = \sum_{i=1}^{n} K(x_i,q) $$
Here, $K$ is a \emph{positive definite kernel} function, typically based on the distance between $x,y$. The most frequent examples include
$$ K(x,q) = \exp(- \|x-y\|_2^2/\lambda^2)\;\;\; K(x,q) = \exp(- \|x-q\|/\lambda) \; \mbox{and}\;\;\; K(x,y) = (1+\|x-q\|/_2^2/\lambda^2)^{-1}$$
where $\lambda$ is a scaling parameter. For simplicity, we assume that $K(x,x) \leq 1$ for all data points. Notice that for any kernel based on distance we have $K(x,x)=1$ exactly for all $x \in \R^d$.

The algorithm we proposed works as follows: Set $\sigma_1 = 1$. For $i=2,\ldots,m$ set $\sigma_i = -\operatorname{sign} (\sum_{j=1}^{i-1}\sigma_j  K(x_j, x_i))$.
\begin{theorem}
The algorithm above achieves a set of signs $\sigma_1,\ldots,\sigma_m$ such that for any $q$,
$$ \frac{1}{m}\sum_{i=1}^m \sigma_i K(x_i,q) \leq \sqrt{m} $$
\end{theorem}

\begin{proof}
For any kernel $K$ there exist a mapping $\phi: \R^d \to {\cal V}$ to an inner product space $\cal V$ such that 
$$ K(x,q) = \ip{\phi(x), \phi(q)} $$
Using this function $\phi$ our objective function becomes
\[
\sum_{i=1}^m \sigma_i K(x_i,q) = \sum_{i=1}^m \sigma_i \ip{\phi(x_i), \phi(q)} =  \ip{ \sum_{i=1}^m \sigma_i \phi(x_i), \phi(q)} \leq  \|\phi(q)\| \cdot \left\|  \sum_{i=1}^m \sigma_i \phi(x_i) \right\| 
\]
Since $\|\phi(q)\| \leq 1$ we reduced the problem to bounding the norm of $ \sum_{i=1}^m \sigma_i \phi(x_i) $.
%
We will show by induction on $i$ that 
$$\left\| \sum_{j=1}^i \sigma_j \phi(x_j) \right\|^2 \le \sum_{j=1}^i \left\|\phi(x_j)\right\|^2 \leq i$$
This is trivially true for $i=1$ since $\|\phi(x)\| \leq 1$. 
Using our induction assumption we get
\begin{eqnarray*}
\left\| \sum_{j=1}^{i}\sigma_j \phi(x_j)\right\|^2 =& \left\|\sum_{j=1}^{i-1}\sigma_j \phi(x_j)\right\|^2 + \|\phi(x_i)\|^2 + 2\ip{ \sum_{j=1}^{i-1}\sigma_j \phi(x_j), \sigma_i \phi(x_i)} \leq & \text{induction assumption}\\
& \sum_{j=1}^{i-1} \|\phi(x_j)\|^2 + \|\phi(x_i)\|^2 + 2\sigma_i \sum_{j=1}^{i-1}\sigma_j K(x_j, x_i) = & \text{choice of } \sigma_i \\
& \sum_{j=1}^{i} \|\phi(x_j)\|^2 - 2|\sum_{j=1}^{i-1}\sigma_j K(x_j, x_i)| \le \sum_{j=1}^{i} \|\phi(x_j)\|^2 \\
\end{eqnarray*}
\end{proof}

Using the framework above this provides a deterministic coreset construction for kernel density estimation of size $O(1/\eps^2)$ such that $\forall \;q\;\; |\tilde f(q) - f(q)| \le \eps n$. This matches, and greatly simplifies, the results achieved by \cite{DBLP:conf/soda/PhillipsT18} and \cite{DBLP:journals/corr/abs-1802-01751}. Along with an $\eps$-net argument, it also gives rise to a streaming algorithm for kernel density estimation with a memory requirement of $O\left(\log\left(d\log(1/\eps)\right)^3/\eps^2\right)$ vectors.

\paragraph{Note} The Theorem above provides an upper bound of $\sqrt{m}$ for the sign discrepancy. One can in fact construct an example where this upper bound is indeed what the algorithm achieves. However, for real data with vectors clustered together, as one should expect when requiring a density estimation algorithm, we expect this algorithm to perform much better than the worst-case bound. We leave it to future work to define properties of the data that ensure better guarantees by our algorithm.


\bibliography{density}


\appendix

\section{Proofs for Section~\ref{sec:sketch}, Sketching Coresets} 

\begin{lemma} \label{lem:compactor}
Let $k$ be an integer and assume we have black box access to a solver that for any $k$ inputs $x_1,\ldots,x_k$ obtains $k$ signs $\sigma_1,\ldots,\sigma_k$ such that
$$\max_q \left| \sum_{i=1}^{k} \sigma_i f(x_i, q)\right| \leq K_k$$
Then there exist a streaming algorithm requiring a memory buffer of $k$ input items that given a stream of length $n$ outputs a stream $z_1,\ldots,z_m$ with the following properties
\begin{itemize}
\item $\{z_i\}_i$ is a subset of $\{x_i\}$
\item $\E[m] = n/2$
\item For $\rho >0$, $\Pr[m \geq (1+\rho)n/2] \leq \exp \left( -O\left(\frac{n\rho^2}{k}\right)\right)$
\item For any fixed query $q$, the error associated with $q$, defined as
$$\text{Err}(q) = 2\sum_{i=1}^m f(z_i,q) - \sum_{i=1}^n f(x_i,q)  $$
Can be decomposed as a sum
$$\text{Err}(q) = \sum_{j=1}^{n/k} \text{Err}_j(q)$$
where the different $\text{Err}_j(q)$ are independent, have mean 0 and $|\text{Err}_j(q)| \leq K_k$
\end{itemize}
\end{lemma}

Note that signs achieving $max_q \left| \sum_{i=1}^{k} \sigma_i f(x_i, q)\right| \leq K_k$ are guarantied to exist by the definition of the Komols Complexity. In fact, the quantity above is potentially much smaller than the Komlos Complexity because it is computed for a specific set of points $x_i$ as opposed to the worst such possible set. 

\begin{proof}
The algorithm operates as follows. It keeps a buffer of $k$ items. Once the buffer fills with $x_1,\ldots,x_k$ it obtains the signs guaranteeing
$$\max_q \left| \sum_i \sigma_i f(x_i, q)\right| \leq K_k$$
Then, the algorithm output to the stream the with probability $1/2$ each data points $\{x_i \; | \; \sigma_i = 1\}$ or $\{x_i \; | \; \sigma_i = -1\}$ with twice the weight.
$$\sum_{i ,\; \sigma_i=1} 2f(x_i, q) = \sum_{i} f(x_i, q) +  \sum_{i} \sigma_i f(x_i, q)$$
$$\sum_{i ,\; \sigma_i=-1} 2f(x_i, q) = \sum_{i} f(x_i, q) - \sum_{i} \sigma_i f(x_i, q)$$

%Now, consider the set $P=\{i:\sigma_i=1\}$ and $N=\{i:\sigma_i=-1\}$.
%$$ \sum_i \sigma_i f(x_i, q) = \sum_i \sigma_i f(x_i, q) + \sum_i f(x_i,q) - \sum_i f(x_i,q) = 2\sum_{i \in P} f(x_i,q) - \sum_i f(x_i, q)   $$
%$$ -\sum_i \sigma_i f(x_i, q) = \sum_i -\sigma_i f(x_i, q) + \sum_i f(x_i,q) - \sum_i f(x_i,q) = 2\sum_{i \in N} f(x_i,q) - \sum_i f(x_i, q)   $$
%Thus, we draw a random coin and either output to the stream the data points of $P$ or $N$. We call this operation a single compaction. 
%\el{this is clunky way to say this...}
The expected output length is $k/2$ for every $k$ inputs, insuring $\E[m]=n/2$. Moreover, the output length is a sum of $n/k$ independent random variables with mean $k/2$ and a maximum value of $k$. Chernoff-Hoeffding's bound can guarantee the stated concentration around $\E[m]$.
As for the error term w.r.t.\ a fixed query $q$, it is clearly the sum of errors accumulated in each compaction. The error incurred in a compaction is either 
$\sum_i \sigma_i f(x_i, q)$ or $-\sum_i \sigma_i f(x_i, q)$ with equal probability. It follows that it is a Radamacher r.v. scaled by a magnitude of at most $K_k$ as required.
\end{proof}

Notice in the above lemma the last property of the streaming algorithm; it ensures that deterministically $|\text{Err}(q)| \leq (K_k/k) n$, and that w.h.p.\ (via Chernoff's inequality) for a fixed $q$ 
$$|\text{Err}(q)| \lesssim K_k \sqrt{n/k} = \sqrt{\frac{K_k^2 n}{k}}$$
This high probability bound will be used below for our streaming algorithm. In high level the idea is to have a sequence of streaming boxes that we call compactors, each receiving an input stream and outputting a stream that is of half the size of the input (in expectation). Eventually we will get to a stream that is small enough to store in memory. The randomized bound above shows that for the compactors that handle a large stream, we can afford to have a very crude sketch because the error that they incur is roughly $\sqrt{n}$ even if all they do is uniformly sample half of the stream. However, for the compactors that observe the shorter streams that high probability bound is no longer as strong and we will have to make use of the fact that $K_k \ll k$.


\begin{proof} [Proof of Theorems~\ref{thm:streaming} and~\ref{thm:streaming2}]
Denote the streaming algorithm of Lemma~\ref{lem:compactor} as a compactor. The streaming algorithm operates as follows. At any given time it maintains a hierarchy of compactors where the hierarchy is measured in levels, starting from 0. The Compactor of level $h$ receives inputs of weight $2^h$ and outputs items of weight $2^{h+1}$. In the beginning we have a single compactor at level $h=0$. Once it outputs items to its output stream we open a compactor at level $h=1$ and so on. After observing $n$ items let $H$ be the level of the final compactor, meaning the compactor that never began an output stream.

The sketch at that point contains all the data points contained in the buffers of the different compactors, weighted according to the level of the compactor. For a query $q$ and hierarchy level $h$ we analyze the error
$$\text{Err}_h(q) = 2^h \sum_{j=1}^{\floor{n_h/k_h} } K_h Y_{ij} \ .$$
Here, $k_h$ is the capacity of the buffers at level $h$, $n_h$ is the length of the stream observed at level $h$. The multiplier of $2^h$ is there since level $h$ observes items of weight $2^h$. The sum over $j$ is over the number of compactions at level $h$. $K_h$ is used to denote the discrepancy of each compaction using a buffer of $k_h$ points. Finally, the $Y_{ij}$ are independent random variables with mean zero and absolute value of at most $1$. The overall error for the query $q$ is the sum over all levels of its errors
$$ \text{Err}(q) = \sum_{h=0}^H \text{Err}_h(q)$$

To achieve a deterministic bound we set all $k_h=k$, and set the compactors to choose the signed set with the least cardinality, ensuring $n_h \leq n_{h-1}/2$ and $H \leq \log_2(n/k)$, thus
$$ \text{Err}(q) \leq (K_k/k) \log_2(n/k) n $$

We are however interested in a high probability bound. Specifically we would like to guarantee for $\delta > 0$ that for any query $q$ the error is bounded by $\eps n$ w.p.\ at least $1-\delta$. To this end we consider a different analysis for the top and bottom layers. For the top layers, the stream can be arbitrarily short so we will use the deterministic analysis as above. For the bottom layers, we can guarantee a minimal length to the stream and obtain tighter bounds.

The bottom layers will be those of height $h=0$ up to $h=H'$ where $H'$ is set adaptively in a way that $H'$ is the maximum integer such that for all $h \leq H'$, 
\begin{equation} \label{eq:Htag_nh}
 \forall h \leq H' , \ \ n_h \geq c_1 k \log^2(n) \log(\log(n)/(c_2 \delta ))
\end{equation}
for some sufficiently large constant $c_1$ and small constant $c_2$ to be determined later.

According to Lemma~\ref{lem:compactor} we have for $ h\leq H'$ that for some appropriate constant $c_1$ in  \eqref{eq:Htag_nh},
$$ \Pr[n_{h+1} \geq (1+1/\log(n))n_{h}/2] \leq \exp\left( -\frac{n_{h}}{k_h \log^2(n)}  \right) \leq c_2\delta/  \log(n) $$
We can now union bound over all 
$$h \leq \log_2\left( n / \left( c_1 k \log^2(n) \log(1/(c_2 \delta\log(n))) \right)\right) + 2 \leq \log_2(n)$$
and conclude that w.p.\ at least $1-\delta/2$
\begin{equation} \label{eq:Htag}
H' \leq \log_2\left( n / \left( c_1 k \log^2(n) \log(1/(c_2 \delta \log(n))) \right)\right) + 2 \leq \log_2(n)
\end{equation}
and
\begin{equation} \label{eq:nh small}
\forall h \leq H', \ \ n_{h} \leq 3 \cdot n/2^{h}
\end{equation}


Let's proceed to bound the error of layer $h$. Since this error is a sum of i.i.d.\ bounded variables of mean zero we can use Chernoff- Hoeffding's inequality and obtain that
$$ \Pr\left[   \left| \text{Err}_h(q) \right| \geq \eps n \right] =$$
$$ \Pr\left[   \left| 2^h \sum_{j=1}^{\floor{n_h/k_h} } K_h Y_{ij} \right| \geq \eps n \right] \leq \exp \left( O\left( -\frac{ \eps^2 n^2}{ 2^{2h} \sum_{j=1}^{\floor{n_h/k_h} } K_h^2}   \right)\right) =$$
$$\exp \left( -O\left(\frac{\eps^2 n^2}{n_h 2^{2h} K_h^2/k_h  } \right) \right) 
$$
We now make use of the the bound on $n_h$ in Equation~\eqref{eq:nh small}. Notice that the random events determining the length $n_h$ do not affect the realization of the random variables used in layer $h$, so we can indeed use the bound on $n_h$ for controlling the error at level $h$. We get that since $n_h \leq 3n/2^h$
$$ \Pr\left[   \left| \text{Err}_h(q) \right| \geq \eps n \right] \leq 
\exp \left( -O\left(\frac{\eps^2 n}{ 2^h K_h^2/k_h  } \right) \right) 
$$
In particular, for $\eps_h = O\left( \sqrt{\frac{2^h K_h^2 \log((H'-h+1)^2/\delta)}{k_h n}} \right)$  it holds that
$$ \Pr\left[   \left| \text{Err}_h(q) \right| \geq \eps n \right] \leq \delta/4(H'-h+1)^2$$
and a union bound ensures that with probability at least $1-\delta$, in addition to Equations~\eqref{eq:Htag} and~\eqref{eq:nh small} we have for all $h \leq H'$ that
$$\left| \text{Err}_h(q) \right| \leq \eps_h n$$
and 
$$\left| \sum_{h=0}^{H'} \text{Err}_h(q) \right| \leq \left(\sum_{h=0}^{H'} \eps_h\right)n $$

Let's proceed to bound the error on the top layers $H'+1$ to $H$. In these top layers we use a deterministic algorithm guaranteeing the stream is cut at least times 2 from layer to layer. We assign all top layers a budget of $k_h=k$ for the compactors. Since $2^{H+1} < n/k$ this gives a bound of 
$$
H-H' \leq \log_2(n/k2^{H'}) = O\left( \log\left( c_1 \log^2(n) \log(1/(c_2 \delta \log(n))) \right)\right) = 
$$
$$
O\left( \log \log(n/\delta) \right)
$$
leading to a bound of 
$$ \left| \sum_{h=H'+1}^H \text{Err}_h(q) \right| \leq O\left( \log \log(n/\delta) (K_k/k) n \right) $$
and an overall bound of
\begin{equation} \label{eq:err total}
|\text{Err}(q)| \leq n \cdot O\left( \log \log(n/\delta)(K_k/k) + \sum_{h=0}^{H'} \eps_h \right)
\end{equation}

We are now left with the task of defining $k_h$ for the lower layers; we do so differently depending on the value of $K_k$ as a function of $k$. Recall that we aim to deal with two settings. In the first $K_k=\min\{\komconstant,k\}$ and in the second $K_k=\min\{\komconstant \sqrt{k}, k\}$. In both cases  $\komconstant$ is independent of $k$. We start by dealing with the first scenario.

We set $k_h = \max\{2, \ceil{(2/3)^{H'-h}k}\}$. By using the definition of $H'$ in Equation~\eqref{eq:Htag} we see that
\begin{align*}
\eps_h = & O\left( \frac{K_k}{k} \sqrt{\frac{2^h K_h^2 k^2 \log((H'-h+1)^2/\delta)}{k_h K_k^2 n}} \right) = & n=\Omega(2^{H'}k\log(H'/\delta)) \\
  	   &  O\left( \frac{K_k}{k} \sqrt{\frac{2^h K_h^2 k }{k_h K_k^2 2^{H'}}} \right) = \\
    	   &  O\left( \frac{K_k}{k} \sqrt{\frac{K_h^2 k }{k_h K_k^2 2^{H'-h}}} \right)
\end{align*}
We move to control the sum over $\eps_h$ by bounding the multiple of $K_k/k$. For $k_h \geq \komconstant$
$$
\frac{K_h^2 k}{2^{H'-h} K_k^2 k_{h}} \leq \frac{\komconstant^2 k}{2^{H'-h} \komconstant^2 (2/3)^{H'-h}k} =
(3/4)^{H'-h}
$$
For $k_h \leq \komconstant$ we must have $h \leq H''$ with $(2/3)^{H'-H''} \leq \komconstant/k$, meaning that $2^{H'-H''} \geq k/\komconstant$. For such $h$ we get
\begin{align*}
 \frac{K_h^2 k}{2^{H'-h} K_k^2 k_{h}} = &  O\left( \frac{ k_h k}{2^{H'-h} \komconstant^2 } \right) =  \\
    	   &  O\left( \frac{ k_h k}{2^{H''-h} k\komconstant } \right) = & k_h \leq \komconstant\\
	   &  O\left( (1/2)^{H''-h}   \right) 
\end{align*}

If follows that 
$$ \sum_{h=0}^{H'} \eps_h = O(K_k/k) = O(\komconstant/k) $$
translating to an overall bound for the error of
$$ |\text{Err}(q)| \leq n \cdot O\left( \log \log(n/\delta)(\komconstant/k)  \right)$$
The overall memory consumption for the bottom layers is $O(k)$ since all layers of size 2 can be implemented jointly by sampling. The top layers require $O(\log\log(n/\delta)k)$ memory. If follows that by setting $k= O\left(\komconstant \log\log(n/\delta) / \eps\right)$ the error is bounded by $\eps n$ w.p. $1-\delta$ and the overall memory requirement is 
$$ O\left( \log \log(n/\delta)^2 (\komconstant/\eps)  \right) $$


We are now ready for the case where $K_k = \min\{k, \komconstant\sqrt{k}\}$. We set $k_h=2$, resulting in $K_h=2$, for all $h \leq H'$ and obtain
\begin{align*}
\eps_h = & O\left( \sqrt{\frac{2^h K_h^2 \log((H'-h+1)^2/\delta)}{k_h n}} \right) = & n=\Omega(2^{H'}k\log(H'/\delta)), \ K_h^2/k_h=O(1) \\
  	   &  O\left( \sqrt{2^{h-H'}/k} \right) 
\end{align*}
and 
$$ \sum_h \eps_h = O\left(1/\sqrt{k}\right) $$
Plugging this to the overall error in Equation~\eqref{eq:err total} we get
$$ \text{Err}(q) = n \cdot O\left( \frac{\log \log(n/\delta)\komconstant + 1}{\sqrt{k}}  \right) =  n \cdot O\left( \frac{\log \log(n/\delta)\komconstant }{\sqrt{k}}  \right) $$

Since layers of $k_h=2$ can all be implemented in constant memory the memory requirement is dominated by the top layers and is in the order of $k \cdot \log\log(n/\delta)$, hence by setting 
$$k \approx \frac{\komconstant^2 \log^2\log(n/\delta)}{\eps^2}$$ 
we get a bound of $\eps n$ for the error and a memory usage of 
$$ O\left( \frac{\komconstant^2 \log^3\log(n/\delta)}{\eps^2}\right) $$
\end{proof}



\end{document}
